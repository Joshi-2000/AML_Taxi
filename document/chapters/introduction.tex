\section{Einleitung (D.R.)}
Im Rahmen des Moduls \textit{Datenerfassung und Datenhaltung 2} wurde ein Projekt mit dem Thema Pegelstände erstellt.
Dieses stellt über eine Webapplikation die aktuellen und historischen Daten der Flüsse und Seen innerhalb Deutschlands dar.
Die Themenwahl begründet sich durch die Aktualität des Klimawandels und den daraus folgenden Gefahren und Risiken. 
Es entstehen häufiger Wetterextreme und damit treten sowohl Überschwemmungen als auch Dürren auf. Mögliche Folgen sind:

\begin{enumerate}
    \item Wassernotstände
    \item Wirtschaftliche Schäden
    \item Gefährdung von Gesundheit und Leben
    \item Brandrisiken
    \item Artensterben
\end{enumerate}~\\
Mit diesem Projekt sollen die oben genannten Folgen verringert werden, indem die über eine Website (\href{https://www.pegelonline.wsv.de/gast/start}{Pegelonline.de}) bezogenen Wasserdaten gesichert und diese in einer Webapplikation bereitgestellt werden. 
Die Datenhaltung bietet sowohl die Möglichkeit einer Risikoanalyse durch eine Auswertung als auch die Möglichkeit einer Vorhersage unter Verwendung verschiedener Lernalgorithmen. 
Die Applikation kann diese Daten anschließend grafisch darstellen und damit der Öffentlichkeit zur Verfügung stellen.



